\documentclass[pdftex,hyperref]{beamer}

\usepackage[spanish]{babel}
\usepackage[T1]{fontenc}
\usepackage[latin9]{inputenc}

\usepackage{graphicx}
\usepackage{url}
\usepackage{hyperref}
\usepackage{amssymb}
\usepackage{colortbl}
\usepackage{listings}
\usepackage{fancyvrb}
\usepackage{multirow}
\usepackage{hyperref}
\usepackage{bibentry}
\usepackage{listings}
\usepackage{tabularx}

\definecolor{linkblue}{RGB}{49,57,174}
\definecolor{dkgreen}{rgb}{0,0.6,0}
\definecolor{gray}{rgb}{0.5,0.5,0.5}
\definecolor{mauve}{rgb}{0.58,0,0.82}

\lstdefinelanguage{scala}{
  morekeywords={abstract,annotation,case,catch,class,def,%
    do,else,extends,false,final,finally,%
    for,if,implicit,import,match,mixin,%
    new,null,object,override,package,%
    private,protected,requires,return,sealed,%
    super,this,throw,trait,true,try,%
    type,val,var,while,with,yield,
    macro},
  sensitive=true,
  morecomment=[l]{//},
  morecomment=[n]{/*}{*/},
  morestring=[b]",
  morestring=[b]',
  morestring=[b]"""
}
\lstset{frame=tb,
  language=scala,
  aboveskip=3mm,
  belowskip=3mm,
  showstringspaces=false,
  columns=flexible,
  basicstyle={\small\ttfamily},
  numbers=none,
  numberstyle=\tiny\color{gray},
  keywordstyle=\color{blue},
  commentstyle=\color{dkgreen},
  stringstyle=\color{mauve},
  frame=single,
  breaklines=true,
  breakatwhitespace=true
  tabsize=3
}

% Configuracion del documento PDF.
\hypersetup{
  pdfcreator=Jes�s L�pez Gonz�lez,
  backref%,
  %%pdfpagemode=FullScreen
}

% Configuracion pagina principal

\title{\textbf{Programaci�n Funcional en Scala}}
\subtitle{\textbf{ -- Tema 3 -- \\ Constructores de Tipos, Type Classes y Funtores }}
\author[Jes�s L�pez Gonz�lez]{Jes�s L�pez Gonz�lez\\jesus.lopez@hablapps.com}
\institute[@jeslg]{Programaci�n Funcional en Scala\\ Habla Computing}
\date{Cursos ETSII-URJC 2015}

% Eleccion estilo de la presentacion

\mode<presentation>
{
 \usetheme{Madrid}
 \setbeamercovered{transparent}
}

\def\newblock{\hskip .11em plus .33em minus .07em}

% Configuracion del logo de la imagen

\subject{Talks}

\pgfdeclareimage[height=0.5cm]{university-logo}{images/logoURJC}
\logo{\pgfuseimage{university-logo}}

\setcounter{tocdepth}{1}

%NOANIMACION
\beamerdefaultoverlayspecification{}

% Volver a recordar tabla de contenidos en subsecciones

\AtBeginSection[]
{
  \begin{frame}<beamer>{�ndice}
    \tableofcontents[currentsection]
  \end{frame}
}

\nobibliography* 

\begin{document}

\begin{frame}
  \titlepage
\end{frame}

\section{Higher Kinded Types}

\begin{frame}[fragile]
  \frametitle{Higher Kinded Types}

  \begin{block}{Constructor de Tipos}
    Un constructor de tipos es un tipo que recibe argumentos tipo para
    construir un nuevo tipo. Ejemplos de constructores de tipo pueden
    ser \emph{List} (recibe un argumento tipo), \emph{Option} (recibe
    un argumento tipo) o \emph{Map} (recibe dos argumentos tipo).
  \end{block}

  %% Option no puede usarse como tal, se requieren par�metros tipo.

  \begin{lstlisting}[language=scala, caption=Usando constructores de tipo]
scala> val opt: Option[Int] = Some(3)
opt: Option[Int] = Some(3)

scala> val lst: List[String] = List("how", "are", "you")
lst: List[String] = List(how, are, you)

scala> val map: Map[Int, String] = Map(1 -> "one", 2 -> "two")
map: Map[Int,String] = Map(1 -> one, 2 -> two)
  \end{lstlisting}
\end{frame}

\begin{frame}[fragile]
  \frametitle{Higher Kinded Types}

  \begin{block}{Kinds}
    \begin{itemize}
      \item Proper: \emph{(*)} Ej: String, Int, List[Int], etc.
      \item First-order: \emph{(* -> *)} Ej: Option[\_], List[\_], etc.
      \item Higher-order: \emph{((* -> *) -> *)} Ej: Functor[F[\_]], etc.
    \end{itemize}
  \end{block}
\end{frame}

\begin{frame}[fragile]
  \frametitle{Higher Kinded Types}

  %% Mostrar un ejemplo con Tuple2

  \begin{lstlisting}[language=scala, caption=Usando constructores de tipo]
scala> def show[C[_, _], A, B](arg: C[A, B]): String = arg.toString
show: [C[_, _], A, B](arg: C[A,B])String

scala> show(Map(1 -> "one", 2 -> "two"))
res0: String = Map(1 -> one, 2 -> two)

scala> show(List(1, 2, 3))
<console>:9: error: inferred kinds of the type arguments (scala.collection.LinearSeqOptimized,Int,List[Int]) do not conform to the expected kinds of the type parameters (type C,type A,type B).
scala.collection.LinearSeqOptimized's type parameters do not match type C's expected parameters:
type Repr's bounds <: scala.collection.LinearSeqOptimized[A,Repr] are stricter than type _'s declared bounds >: Nothing <: Any
              show(List(1, 2, 3))
              ^
<console>:9: error: type mismatch;
 found   : List[Int]
 required: C[A,B]
              show(List(1, 2, 3))
                       ^
  \end{lstlisting}
\end{frame}

\section{Implicits}

\section{Type Classes}

\section{Funtores}

\section{Takeaways}

\end{document}
